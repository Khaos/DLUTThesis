%% -*- Mode: LaTeX Memoir; tab-width: 4;

%% DLUTThesis.tex    The Main File for DLUT Doctor Thesis
%%
%% Created by Dazhi Jiang, 2016-05-27 13:15:22 +0800 (Fri, 27 May 2016)
%% Copyright (c) 2016 Dazhi Jiang All Rights Reserved.

%% TextMate Settings
% !TEX TS-program = xelatex
% !TEX encoding = UTF-8 Unicode
% !TEX TS-options = -shell-escape -interaction=nonstopmode -file-line-error

%% TeXstudio/ATOM Settings
% !TEX encoding = utf8
% !TEX program = xelatex
% !TEX root = DLUTThesis.tex


\documentclass[a4paper, twoside, openany, extrafontsizes]{dlutthesis}

\usepackage{lipsum}

\dlutset{%
	%% - Title page
	titlepagestyle=dlut,
	%% - Chapter heading
	chapterheading=dlut,
	%% - Section heading
	sectionheading=dlut,
	%% - Section numbering
	maxsecnumdepth=subsection,
	%% - Pgf/TikZ
	pgf/externalmainfile=DLUTThesis,
	pgf/externalprefix={Figs/pdf/pgf-}
}

%% * Define the elements of title page
\dlutset{%
	thesis={博~~士~~学~~位~~论~~文},
    title={大连理工大学博士学位论文模版},
    etitle={The DLUT PhD Thesis Template},
    author={中~~ ~~文},
    authorid={3.14159265},
	major={文~学},
    supervisor={中~~ ~~文 教授},
	date={\today},
	% university={Dalian University of Technology},
}

\usepackage{mathtools}
\usepackage{esvect}

\usetikzlibrary{circuits,%
	circuits.logic,%
	intersections,%
	scopes, arrows, fadings, patterns,%
	decorations.pathreplacing,%
	decorations.pathmorphing,%
	positioning,
	shadows,
	shadings,%
	shapes.misc,%
	shapes.geometric,%
	external,
	calc,
	mindmap,
	chains,
}

\pgfdeclarelayer{background}
\pgfdeclarelayer{foreground}
\pgfsetlayers{background,main,foreground}


\begin{document}

%% * The title page
\maketitle

%% * Plagiarism declaration
\cleardoublepage
\makedeclaration

%% * The front matter
\frontmatter

%%   - Chinese abstract
\cleardoublepage
\chapter*{摘  要}

中文

第二行


%%   - English abstract
\cleardoublepage
\chapter*{ABSTRACT}

English

Line 2


%%   - ToC, LoF and LoT
%%     - Chinese ToC
\cleardoublepage
\tableofcontents*

%%     - English ToC
\cleardoublepage
\tableofencontents*

%%     - List of figures
\cleardoublepage
\listoffigures*

%%     - List of tables
\vspace{5.2em}
\listoftables*

%% * The main matter
\mainmatter

%%   - Chapter I
\chapter{章节标题}{Headings}

道可道,非常道。名可名,非常名。无名天地之始。有名万物之母。故常无欲以观其妙。常有欲以观
其徼。此两者同出而异名,同谓之玄。玄之又玄,众妙之门。

天下皆知美之为美,斯恶矣;皆知善之为善,斯不善已。故有无相生,难易相成,长短相形,高下相
倾,音声相和,前後相随。是以圣人处无为之事,行不言之教。万物作焉而不辞。生而不有,为而不恃,
功成而弗居。夫唯弗居,是以不去。

\section{节}{Section Headings}
\label{sec:section}


不尚贤, 使民不争。不贵难得之货,使民不为盗。不见可欲,使民心不乱。是以圣人之治,虚其心,
实其腹,弱其志,强其骨;常使民无知、无欲,使夫智者不敢为也。为无为,则无不治。

道冲而用之,或不盈。渊兮似万物之宗。解其纷,和其光,同其尘,湛兮似或存。吾不知谁之子,象
帝之先。

\subsection{小节}{Subsection Headings}
\label{sub:subsection}

天地不仁,以万物为刍狗。圣人不仁,以百姓为刍狗。天地之间,其犹橐迭乎?虚而不屈,动而愈出
。多言数穷,不如守中。

谷神不死是谓玄牝。玄牝之门是谓天地根。绵绵若存,用之不勤。

\subsubsection{小小节}{Subsubsection Headings}
\label{subs:subsubsection}

天长地久。天地所以能长且久者,以其不自生,故能长生。是以圣人後其身而身先,外其身而身存。
非以其无私邪!故能成其私。

上善若水。水善利万物而不争,处众人之所恶,故几於道。居善地,心善渊,与善仁,言善信,正善
治,事善能,动善时。夫唯不争,故无尤。

\chapter{图表}{Floats}
\label{chap:floats}

\section{图}{Figures}
\label{sec:figures}

\begin{figure}[!htbp]
	\centering
	EXAMPLE FIGURE WITH BITWONUMCAPTION
	\bicaption[fig:bi1]%
		{}{长\cs{bicaption}}%
		{Figure}{Long \cs{bicaption}}
\end{figure}

\begin{figure}[!htbp]
	\centering
	% \dlutset{pgf/makeexternalfig}
	\dlutset{pgf/makenextfig=c-IP}
	%% -*- Mode: LaTeX Memoir; tab-width: 4;

%% InvertedPendulum.tikz    Inverted Pendulum
%% Draw an inverted pendulum using PGF/TikZ
%% MAIN FILE: ../../TikZGallery.tex
%% Created by Dazhi Jiang, 2011-02-01 16:49:22 +0000 (Tue,  1 Feb 2011)
%% Copyright (c) 2011 Dazhi Jiang. All Rights Reserved. 

%% TextMate Settings
%!TEX root = ../../TikZGallery.tex

\newif\ifpendulumcolored
\pendulumcoloredtrue

\ifpendulumcolored

\tikzset{%
	interface/.style={
		% The border decoration is a path replacing decorator. 
		% For the interface style we want to draw the original path.
		% The postaction option is therefore used to ensure that the
		% border decoration is drawn *after* the original path.
		postaction={draw, decorate, decoration={border, angle=-45,
					amplitude=0.3cm, segment length=2mm}}},
	helparrow/.style={>=latex', blue, thick},
	% helparrow/.style={>=latex', draw=blue, fill=blue, very thick},
	helpline/.style={thin, black!90, opacity=0.5},
	force/.style={>=stealth, draw=blue, fill=blue, ultra thick},
	cart/.style={draw = black,%fill = black!40,%
				top color = green!5,%
				bottom color = green!40,%
				% pattern=horizontal lines gray,%
				},
	pendulum/.style={draw = black,% fill = black!25,%
					left color = red!10,%
					bottom color = red!40,%
					% pattern=horizontal lines gray,%
					},
	ground/.style={fill=brown!20},
	inner wheel/.style={fill=Silver!50},
	outer wheel/.style={thin,double = black!75,double distance = 0.6mm,black},
	wheel shadow/.style={thin,fill = black!70,path fading = south},
	xycoord/.style={<->, thick, black!50},
}

\else

\tikzset{%
	interface/.style={
		% The border decoration is a path replacing decorator. 
		% For the interface style we want to draw the original path.
		% The postaction option is therefore used to ensure that the
		% border decoration is drawn *after* the original path.
		postaction={draw, decorate, decoration={border, angle=-45,
					amplitude=0.3cm, segment length=2mm}}},
	helparrow/.style={>=latex', blue, thick},
	% helparrow/.style={>=latex', draw=blue, fill=blue, very thick},
	helpline/.style={thin, black!90, opacity=0.5},
	force/.style={>=stealth, draw=blue, fill=blue, ultra thick},
	cart/.style={draw = black!80,%fill = black!40,%
				top color = black!10,%
				bottom color = black!40,%
				% pattern=horizontal lines gray,%
				},
	pendulum/.style={draw = black!80,% fill = black!25,%
					left color = black!10,%
					bottom color = black!40,%
					% pattern=horizontal lines gray,%
					},
	ground/.style={fill=black!20},
	% ground/.style={fill=black!20},
	inner wheel/.style={fill=black!30},
	outer wheel/.style={thin,double = black!20,double distance = 0.5mm},
	wheel shadow/.style={thin,fill = black!70,path fading = south},
	xycoord/.style={<->, thick, black!50},
}

\fi

\def\ground{%
	\fill [ground] (0, 0) rectangle (49mm, -13mm);
	\fill [ground] (49mm, 0) rectangle (80mm, -5mm);
	\draw [thick, black!80, interface] (0, 0) -- (80mm, 0);
	\draw [thick, black!80] (49mm, 0) -- +(0, -13mm);
	% \draw [|->, helparrow] (49mm, -9mm) -- ++(20mm, 0) node [right] {$\gls{x}$};
	\draw [|->, helparrow] (49mm, -9mm) -- ++(20mm, 0) node [right] {$x$};
}

\def\cart{
	\filldraw [cart] (0,0) rectangle (30mm, 10mm);
	% \node at (15mm, 5mm) {$\gls{M}$};
	% \draw[->, force] (-15mm, 5mm) node [above] {$\gls{F}$} -- (0, 5mm);
	\node at (15mm, 5mm) {$M$};
	\draw[->, force] (-15mm, 5mm) node [above] {$\vv{F}$} -- (0, 5mm);
	% \draw[->, force,] (40mm, -2.5mm) node [above, text = blue] {${\gls{b}}\textcolor{blue}{\dot{\gls{x}}}$} -- (26.5mm, -2.5mm);
	% \draw[->, force,] (40mm, -2.5mm) node [above, text = blue] {$\gls{b}\gls{xspeed}$} -- (26.5mm, -2.5mm);
	\draw[->, force,] (40mm, -2.5mm) node [above, text = blue] {$b\dot{x}$} -- (26.5mm, -2.5mm);
	% \draw[->, force] (-15mm, 5mm) node [above] {$\vv{\gls{F}}$} -- (0, 5mm);
	% \draw[->, force] (-15mm, 5mm) node [above] {$\vv{F}$} -- (0, 5mm);
}

\def\pendulum{%
	\filldraw [pendulum] (-0.8mm, 0) rectangle (0.8mm, 20mm);
}

\def\joint{%
	\filldraw [%thick,%
		draw = black!80,%
		fill = white,%
		% pattern=horizontal lines gray,%
		] (0, 0) circle (1mm);
}

\def\wheel{%
	\fill [wheel shadow] (0, 0) circle (1.75mm);
	\begin{scope}
		\clip (0, 0) circle (1.75mm);
		\fill [inner wheel] (0, -1mm) circle (2mm);
	\end{scope}
	\fill [fill=black!90] (0, 0) circle (0.5mm);
	\draw [outer wheel] (0, 0) circle (2mm);
}

\def\xycoord{%
	\draw [xycoord] (8mm,0) node [below right] {$x$} -- (0,0) -- (0,8mm) node [above left] {$y$};
}

%\tikzsetnextfilename{PIDBode}
\begin{tikzpicture}[> = latex',%
		scale = 1,%
		text = blue,%
		execute at end picture = {
			\begin{pgfonlayer}{background}
				%% * Compute a few help coordinates
				% \coordinate (bl) at (-40mm, -24mm);
				% \coordinate (tr) at (58mm, 45mm);
				\coordinate (bl) at ($(current bounding box.south west) + (-5mm, -5mm)$);
				\coordinate (tr) at ($(current bounding box.north east) + (+5mm, +5mm)$);
				\draw [very thin, fill=white, %
					  rounded corners = 1.5pt, %
					  draw=black!50, %
					  % decorate, %
					  % decoration={random steps,segment length=3pt,amplitude=1pt}, %
					  drop shadow={fill=black!30}, %
					  % general shadow={fill=black!10, shadow scale=1.01}, %
					  % dashed, %
					  ]
						% (inner box.south west) + (-5mm, -5mm) rectangle 
						% (inner box.north east) + (5mm, 5mm);
						(bl) rectangle (tr);
				\useasboundingbox (bl) + (-1mm, -1mm) -- (tr) + (1mm, 1mm);
			\end{pgfonlayer}
		}
		]
	\begin{scope}[local bounding box = inner box]
		\begin{scope}
			\wheel
		\end{scope}
		\begin{scope}[xshift=18mm]
			\wheel
		\end{scope}
		\begin{scope}[xshift=-31mm, yshift=-2.4mm]
			\ground
		\end{scope}
		\begin{scope}[shift = {(-6mm, 2.4mm)}]
			\cart
		\end{scope}
		\begin{scope}[shift = {(9mm, 12.4mm)}]
			\draw [helpline] (0, -20mm) -- (0, 25mm);
			\draw [helpline] (110:20.5mm) -- (110:25mm);
			\draw [|->, helparrow] (0, 23mm) 
				arc [radius=23mm, start angle=90, delta angle=20] ;
			% \node at (97:25mm) {$\gls{thetaa}$};
			\node at (97:25mm) {$\theta'$};
			\draw [|->, helparrow] (0, -9mm) 
				arc [radius=9mm, start angle=-90, delta angle=195] ;
			% \node [above right] at (50:8.5mm) {$\gls{theta}$};
			\node [above right] at (50:8.5mm) {$\theta$};
			\draw [thin, helpline] (110:14mm) -- (-10mm, 10mm) 
				node [left,opacity=1] {{\color{blue}$m$, $I$}};
				% node [left] {$\gls{m}$, $\gls{I}$};
		\end{scope}
		\begin{scope}[shift = {(9mm, 12.4mm)}, rotate=20]
			\pendulum
		\end{scope}
		\begin{scope}[shift = {(9mm, 12.4mm)}]
			\joint
		\end{scope}
		\begin{scope}[shift = {(35mm, 12.4mm)}]
			\xycoord
		\end{scope}
	\end{scope}

\end{tikzpicture}

	\caption{Inverted Pendulum.}
	\label{fig:IP}
\end{figure}


\chapter{中文}{Chinese} % start of your main text


道可道,非常道。名可名,非常名。无名天地之始。有名万物之母。故常无欲以观其妙。常有欲以观
其徼。此两者同出而异名,同谓之玄。玄之又玄,众妙之门。

天下皆知美之为美,斯恶矣;皆知善之为善,斯不善已。故有无相生,难易相成,长短相形,高下相
倾,音声相和,前後相随。是以圣人处无为之事,行不言之教。万物作焉而不辞。生而不有,为而不恃,
功成而弗居。夫唯弗居,是以不去。

不尚贤, 使民不争。不贵难得之货,使民不为盗。不见可欲,使民心不乱。是以圣人之治,虚其心,
实其腹,弱其志,强其骨;常使民无知、无欲,使夫智者不敢为也。为无为,则无不治。

道冲而用之,或不盈。渊兮似万物之宗。解其纷,和其光,同其尘,湛兮似或存。吾不知谁之子,象
帝之先。

天地不仁,以万物为刍狗。圣人不仁,以百姓为刍狗。天地之间,其犹橐迭乎?虚而不屈,动而愈出
。多言数穷,不如守中。

谷神不死是谓玄牝。玄牝之门是谓天地根。绵绵若存,用之不勤。

天长地久。天地所以能长且久者,以其不自生,故能长生。是以圣人後其身而身先,外其身而身存。
非以其无私邪!故能成其私。

上善若水。水善利万物而不争,处众人之所恶,故几於道。居善地,心善渊,与善仁,言善信,正善
治,事善能,动善时。夫唯不争,故无尤。

持而盈之不如其己;揣而锐之不可长保;金玉满堂莫之能守;富贵而骄,自遗其咎。功遂身退,天之
道。

载营魄抱一,能无离乎?专气致柔,能如婴儿乎?涤除玄览,能无疵乎?爱国治民,能无为乎?天门
开阖,能为雌乎?明白四达,能无知乎。

三十幅共一毂,当其无,有车之用。埏埴以为器,当其无,有器之用。凿户牖以为室,当其无,有室
之用。故有之以为利,无之以为用。

五色令人目盲,五音令人耳聋,五味令人口爽,驰骋畋猎令人心发狂,难得之货令人行妨。是以圣人
,为腹不为目,故去彼取此。

宠辱若惊,贵大患若身。何谓宠辱若惊?宠为下。得之若惊失之若惊是谓宠辱若惊。何谓贵大患若身
?吾所以有大患者,为吾有身,及吾无身,吾有何患。故贵以身为天下,若可寄天下。爱以身为天下,若
可托天下。

视之不见名曰夷。听之不闻名曰希。抟之不得名曰微。此三者不可致诘,故混而为一。其上不皦
(jiǎo),其下不昧,绳绳不可名,复归於无物。是谓无状之状,无物之象,是谓惚恍。迎之不见其首,随
之不见其後。执古之道以御今之有。能知古始,是谓道纪。

古之善为士者,微妙玄通,深不可识。夫唯不可识,故强为之容。豫兮若冬涉川;犹兮若畏四邻;俨
兮其若容;涣兮若冰之将释;敦兮其若朴;旷兮其若谷;混兮其若浊;澹兮其若海;飉(liáo,风的声音)兮
若无止。孰能浊以静之徐清。孰能安以动之徐生。保此道者不欲盈。夫唯不盈故能蔽而新成。

致虚极守静笃。万物并作,吾以观复。夫物芸芸各复归其根。归根曰静,是谓复命;复命曰常,知常
曰明。不知常,妄作凶。知常容,容乃公,公乃全,全乃天,天乃道,道乃久,没身不殆。

太上,下知有之。其次,亲而誉之。其次,畏之。其次,侮之。信不足焉,有不信焉。悠兮其贵言,
功成事遂,百姓皆谓∶我自然。

大道废有仁义;慧智出有大伪;六亲不和有孝慈;国家昏乱有忠臣。

绝圣弃智,民利百倍;绝仁弃义,民复孝慈;绝巧弃利,盗贼无有;此三者,以为文不足。故令有所
属,见素抱朴少私寡欲。

绝学无忧,唯之与阿,相去几何?善之与恶,相去若何?人之所畏,不可不畏。荒兮其未央哉!众人
熙熙如享太牢、如春登台。我独泊兮其未兆,如婴儿之未孩;儡儡(lěi,羸弱)兮若无所归。众人皆有馀,而
我独若遗。我愚人之心也哉!沌沌兮。俗人昭昭,我独昏昏;俗人察察,我独闷闷。众人皆有以,而我独
顽且鄙。我独异於人,而贵食母。

孔德之容惟道是从。道之为物惟恍惟惚。惚兮恍兮其中有象。恍兮惚兮其中有物。窈兮冥兮其中有精
。其精甚真。其中有信。自古及今,其名不去以阅众甫。吾何以知众甫之状哉!以此。

曲则全,枉则直,洼则盈,敝则新少则得,多则惑。是以圣人抱一为天下式。不自见故明;不自是故
彰;不自伐故有功;不自矜故长;夫唯不争,故天下莫能与之争。古之所谓∶曲则全者」岂虚言哉!诚
全而归之。

希言自然。故飘风不终朝,骤雨不终日。孰为此者?天地。天地尚不能久,而况於人乎?故从事於道
者,同於道。德者同於德。失者同於失。同於道者道亦乐得之;同於德者德亦乐得之;同於失者失於乐得
之信不足焉有不信焉。

企者不立;跨者不行。自见者不明;自是者不彰。自伐者无功;自矜者不长。其在道也曰∶馀食赘形
。物或恶之,故有道者不处。

有物混成先天地生。寂兮寥兮独立不改,周行而不殆,可以为天下母。吾不知其名,强字之曰道。强
为之名曰大。大曰逝,逝曰远,远曰反。故道大、天大、地大、人亦大。域中有大,而人居其一焉。人法
地,地法天,天法道,道法自然。

重为轻根,静为躁君。是以君子终日行不离轻重。虽有荣观燕处超然。奈何万乘之主而以身轻天下。
轻则失根,躁则失君。

善行无辙迹。善言无瑕谪。善数不用筹策。善闭无关楗而不可开。善结无绳约而不可解。是以圣人常
善救人,故无弃人。常善救物,故无弃物。是谓袭明。故善人者不善人之师。不善人者善人之资。不贵其
师、不爱其资,虽智大迷,是谓要妙。

知其雄,守其雌,为天下溪。为天下溪,常德不离,复归於婴儿。知其白,守其黑,为天下式。为天
下式,常德不忒,复归於无极。知其荣,守其辱,为天下谷。为天下谷,常德乃足,复归於朴。朴散则为
器,圣人用之则为官长。故大制不割。

将欲取天下而为之,吾见其不得已。天下神器,不可为也,为者败之,执者失之。夫物或行或随、或
觑或吹、或强或羸、或挫或隳。是以圣人去甚、去奢、去泰。

以道佐人主者,不以兵强天下。其事好还。师之所处荆棘生焉。军之後必有凶年。善有果而已,不敢
以取强。果而勿矜。果而勿伐。果而勿骄。果而不得已。果而勿强。物壮则老,是谓不道,不道早已。
夫佳兵者不祥之器,物或恶之,故有道者不处。君子居则贵左,用兵则贵右。兵者不祥之器,非君子
之器,不得已而用之,恬淡为上。胜而不美,而美之者,是乐杀人。夫乐杀人者,则不可得志於天下矣。
吉事尚左,凶事尚右。偏将军居左,上将军居右。言以丧礼处之。杀人之众,以悲哀泣之,战胜以丧礼处
之。

道常无名。朴虽小天下莫能臣也。侯王若能守之,万物将自宾。天地相合以降甘露,民莫之令而自均
。始制有名,名亦既有,夫亦将知止,知止可以不殆。譬道之在天下,犹川谷之於江海。

知人者智,自知者明。胜人者有力,自胜者强。知足者富。强行者有志。不失其所者久。死而不亡者
,寿。

大道泛兮,其可左右。万物恃之以生而不辞,功成而不名有。衣养万物而不为主,常无欲可名於小。
万物归焉,而不为主,可名为大。以其终不自为大,故能成其大。

执大象天下往。往而不害安平太。乐与饵,过客止。道之出口淡乎其无味。视之不足见。听之不足闻
。用之不足既。

将欲歙之,必固张之。将欲弱之,必固强之。将欲废之,必固兴之。将欲取之,必固与之。是谓微明
。柔弱胜刚强。鱼不可脱於渊,国之利器不可以示人。

道常无为,而无不为。侯王若能守之,万物将自化。化而欲作,吾将镇之以无名之朴。无名之朴,夫
亦将无欲。不欲以静,天下将自定。

上德不德是以有德。下德不失德是以无德。上德无为而无以为。下德无为而有以为。上仁为之而无以
为。上义为之而有以为。上礼为之而莫之以应,则攘臂而扔之。故失道而後德。失德而後仁。失仁而後义
。失义而後礼。夫礼者忠信之薄而乱之首。前识者,道之华而愚之始。是以大丈夫,处其厚不居其薄。处
其实,不居其华。故去彼取此。

昔之得一者。天得一以清。地得一以宁。神得一以灵。谷得一以盈。万物得一以生。侯王得一以为天
下贞。其致之。天无以清将恐裂。地无以宁将恐废。神无以灵将恐歇。谷无以盈将恐竭。万物无以生将恐
灭。侯王无以贞将恐蹶。故贵以贱为本,高以下为基。是以侯王自称孤、寡、不谷。此非以贱为本邪?非
乎。至誉无誉。不欲琭琭如玉,珞珞如石。

反者道之动。弱者道之用。天下万物生於有,有生於无。

\section{Englsih}{English}
\label{sec:englsih}

\lipsum[1-9]


%% if endnotes then
\printpagenotes

%% if a bibliography then
\begin{thebibliography}
	...
\end{thebibliography}

%% if appendices, then
\appendix
% \chapterstyle{dlutappendix}
\chapter{附录1}{Example 1}
...

\backmatter
%% if Biographical sketch then
% \begin{biosketch} ... \end{biosketch}

\end{document}
