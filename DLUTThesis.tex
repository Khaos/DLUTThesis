%% -*- Mode: LaTeX Memoir; tab-width: 4;

%% DLUTThesis.tex    The Main File for DLUT Doctor Thesis
%%
%% Created by Dazhi Jiang, 2016-05-27 13:15:22 +0800 (Fri, 27 May 2016)
%% Copyright (c) 2016 Dazhi Jiang All Rights Reserved.

%% TextMate Settings
% !TEX TS-program = xelatex
% !TEX encoding = UTF-8 Unicode
% !TEX TS-options = -shell-escape -interaction=nonstopmode -file-line-error

%% TeXstudio/ATOM Settings
% !TEX encoding = utf8
% !TEX program = xelatex
% !TEX root = DLUTThesis.tex


\documentclass[11pt,a4paper,extrafontsizes,oldfontcommands,twoside]{dlutthesis}

\usepackage{lipsum}

\begin{document}
\maxtocdepth{subsection} % put 3 levels into the ToC
\frontmatter
% \thetitlepage
% \approvalpage
% \asuabstract
abstract text
%% if you have any acknowledgements, then
% \asuacknowledgements
acknowledgements text
% \asudedication{ text } % if you want a dedication
\tableofcontents
% \listoffigures % if you have any figures
% \listoftables % if you have any tables
%%% if you have more prelim sections, then
%%% \clearpage
%%% \pagestyle{plain}
%%% \prelimtitle{title} text % for each section before main text
\mainmatter
% \pagestyle{asu}
\chapter{中文} % start of your main text

道可道,非常道。名可名,非常名。无名天地之始。有名万物之母。故常无欲以观其妙。常有欲以观
其徼。此两者同出而异名,同谓之玄。玄之又玄,众妙之门。

天下皆知美之为美,斯恶矣;皆知善之为善,斯不善已。故有无相生,难易相成,长短相形,高下相
倾,音声相和,前後相随。是以圣人处无为之事,行不言之教。万物作焉而不辞。生而不有,为而不恃,
功成而弗居。夫唯弗居,是以不去。

不尚贤, 使民不争。不贵难得之货,使民不为盗。不见可欲,使民心不乱。是以圣人之治,虚其心,
实其腹,弱其志,强其骨;常使民无知、无欲,使夫智者不敢为也。为无为,则无不治。

道冲而用之,或不盈。渊兮似万物之宗。解其纷,和其光,同其尘,湛兮似或存。吾不知谁之子,象
帝之先。

天地不仁,以万物为刍狗。圣人不仁,以百姓为刍狗。天地之间,其犹橐迭乎?虚而不屈,动而愈出
。多言数穷,不如守中。

谷神不死是谓玄牝。玄牝之门是谓天地根。绵绵若存,用之不勤。

天长地久。天地所以能长且久者,以其不自生,故能长生。是以圣人後其身而身先,外其身而身存。
非以其无私邪!故能成其私。

上善若水。水善利万物而不争,处众人之所恶,故几於道。居善地,心善渊,与善仁,言善信,正善
治,事善能,动善时。夫唯不争,故无尤。

持而盈之不如其己;揣而锐之不可长保;金玉满堂莫之能守;富贵而骄,自遗其咎。功遂身退,天之
道。

载营魄抱一,能无离乎?专气致柔,能如婴儿乎?涤除玄览,能无疵乎?爱国治民,能无为乎?天门
开阖,能为雌乎?明白四达,能无知乎。

三十幅共一毂,当其无,有车之用。埏埴以为器,当其无,有器之用。凿户牖以为室,当其无,有室
之用。故有之以为利,无之以为用。

五色令人目盲,五音令人耳聋,五味令人口爽,驰骋畋猎令人心发狂,难得之货令人行妨。是以圣人
,为腹不为目,故去彼取此。

宠辱若惊,贵大患若身。何谓宠辱若惊?宠为下。得之若惊失之若惊是谓宠辱若惊。何谓贵大患若身
?吾所以有大患者,为吾有身,及吾无身,吾有何患。故贵以身为天下,若可寄天下。爱以身为天下,若
可托天下。

视之不见名曰夷。听之不闻名曰希。抟之不得名曰微。此三者不可致诘,故混而为一。其上不皦
(jiǎo),其下不昧,绳绳不可名,复归於无物。是谓无状之状,无物之象,是谓惚恍。迎之不见其首,随
之不见其後。执古之道以御今之有。能知古始,是谓道纪。

古之善为士者,微妙玄通,深不可识。夫唯不可识,故强为之容。豫兮若冬涉川;犹兮若畏四邻;俨
兮其若容;涣兮若冰之将释;敦兮其若朴;旷兮其若谷;混兮其若浊;澹兮其若海;飉(liáo,风的声音)兮
若无止。孰能浊以静之徐清。孰能安以动之徐生。保此道者不欲盈。夫唯不盈故能蔽而新成。

致虚极守静笃。万物并作,吾以观复。夫物芸芸各复归其根。归根曰静,是谓复命;复命曰常,知常
曰明。不知常,妄作凶。知常容,容乃公,公乃全,全乃天,天乃道,道乃久,没身不殆。

太上,下知有之。其次,亲而誉之。其次,畏之。其次,侮之。信不足焉,有不信焉。悠兮其贵言,
功成事遂,百姓皆谓∶我自然。

大道废有仁义;慧智出有大伪;六亲不和有孝慈;国家昏乱有忠臣。

绝圣弃智,民利百倍;绝仁弃义,民复孝慈;绝巧弃利,盗贼无有;此三者,以为文不足。故令有所
属,见素抱朴少私寡欲。

绝学无忧,唯之与阿,相去几何?善之与恶,相去若何?人之所畏,不可不畏。荒兮其未央哉!众人
熙熙如享太牢、如春登台。我独泊兮其未兆,如婴儿之未孩;儡儡(lěi,羸弱)兮若无所归。众人皆有馀,而
我独若遗。我愚人之心也哉!沌沌兮。俗人昭昭,我独昏昏;俗人察察,我独闷闷。众人皆有以,而我独
顽且鄙。我独异於人,而贵食母。

\section{Englsih}
\label{sec:englsih}

\lipsum[1-9]

%% if endnotes then
\printpagenotes
%% if a bibliography then
\begin{thebibliography}...\end{thebibliography}
%% if appendices, then
\appendix
\chapter{...}
...
%% if Biographical sketch then
% \begin{biosketch} ... \end{biosketch}

\end{document}
